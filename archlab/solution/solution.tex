\documentclass{article}

\usepackage{listings}
\usepackage{xcolor}
\usepackage{caption}
\usepackage[a4paper, total={6in, 10in}]{geometry}

\lstset{
  language=C,               
  numbers=left,             
  stepnumber=1,             
  numbersep=10pt,           
  backgroundcolor=\color{white},
  showspaces=false,             
  showtabs=false,               
  tabsize=2,                    
  captionpos=b,                 
  breaklines=true,              
  breakatwhitespace=true,       
  belowcaptionskip=1\baselineskip,
  breaklines=true,
  xleftmargin=\parindent,
  showstringspaces=false,
  basicstyle=\footnotesize\ttfamily,
  keywordstyle=\bfseries\color{blue!40!black},
  commentstyle=\itshape\color{green!40!black},
  stringstyle=\color{orange},
}

\begin{document}
\section{Part A}
\subsection{sum.ys Iteratively sum linked list elements}
\paragraph{Solution}
The struct \textsf{ELE} used in the \textsf{sum\_list()} function shown on listing \ref{lst:ELE-c}.
\lstinputlisting[caption={ELE.c}, label={lst:ELE-c}]{../listings/ELE.c}
The function \textsf{sum\_list()} written is C displayed on listing \ref{lst:sum_list-c}.
\lstinputlisting[caption={sum\_list.c}, label={lst:sum_list-c}]{../listings/sum_list.c}
Full example of sum.ys with stack initialization, test data, and main function depicted on lising \ref{lst:sum-ys} \ref{lst:sum-ys}.
\lstinputlisting[caption={sum.ys}, label={lst:sum-ys}, language={[x86masm]Assembler}]{../listings/sum_list.asm}
\subsection{rsum.ys Recursively sum linked list elements}
\paragraph{Solution}
The same \textsf{ELE} struct from listing \ref{lst:ELE-c} used int the \textsf{rsum\_list()} function.
The C version of \textsf{rsum\_list()} is on listing \ref{lst:rsum_list-c}.
\lstinputlisting[caption={rsum\_list.c}, label={lst:rsum_list-c}]{../listings/rsum_list.c}
Full version of rsum.ys is on listing \ref{lst:rsum-ys}.
\lstinputlisting[caption={rsum.ys}, label={lst:rsum-ys}, language={[x86masm]Assembler}]{../listings/rsum_list.asm}
\newpage
\subsection{copy.ys Copy a source block to a destination block}
\paragraph{Solution}
The C version of \textsf{copy\_block()} is on listing \ref{lst:copy-block-c}.
\lstinputlisting[caption={copy\_block.c}, label={lst:copy-block-c}]{../listings/copy_block.c}
Final version of copy.ys is on listing \ref{lst:copy-block-ys}.
\lstinputlisting[caption={copy.ys}, label={lst:copy-block-ys}, language={[x86masm]Assembler}]{../listings/copy_block.asm}
\newpage
\section{Part B}
\paragraph{Solution}
The updated hcl description of SEQ control signals for implementing IADDQ instruction show on listing \ref{lst:iaddq-seq-hcl}.
\lstinputlisting[caption={Update SEQ contorl signals for IADDQ instuction}, label={lst:iaddq-seq-hcl}, language={Verilog}]{../listings/seq-full.hcl}
\section{Part C}
\paragraph{Solution}
The optimized version of function \textsf{ncopy()} is shown on \ref{lst:ncopy-asm}. 
Average \textsf{CPE} after optimization is \textbf{7.98}.
\lstinputlisting[caption={Optimized version of ncopy.asm}, label={lst:ncopy-asm}, language={[x86masm]Assembler}]{../listings/ncopy.asm}
\end{document}

