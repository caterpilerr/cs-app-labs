\documentclass{article}

\usepackage{listings}
\usepackage{xcolor}
\usepackage{caption}
\usepackage[a4paper, total={6in, 10in}]{geometry}
\usepackage[utf8x]{inputenc}
\usepackage[framemethod=tikz]{mdframed}

\lstset{
  language=C,               
  numbers=left,             
  stepnumber=1,             
  numbersep=10pt,           
  backgroundcolor=\color{white},
  showspaces=false,             
  showtabs=false,               
  tabsize=2,                    
  captionpos=b,                 
  breaklines=true,              
  breakatwhitespace=true,       
  belowcaptionskip=1\baselineskip,
  breaklines=true,
  xleftmargin=\parindent,
  showstringspaces=false,
  basicstyle=\footnotesize\ttfamily,
  keywordstyle=\bfseries\color{blue!40!black},
  commentstyle=\itshape\color{green!40!black},
  stringstyle=\color{orange},
}

\newcommand\mylstcaption{}

\mdfdefinestyle{mymdstyle}{
hidealllines=true,
middleextra={
  \node[anchor=west] at (O|-P)
    {\lstlistingname~\thelstlisting\  (Cont.):~\mylstcaption};},
secondextra={
  \node[anchor=west] at (O|-P)
    {\lstlistingname~\thelstlisting\  (Cont.):~\mylstcaption};},
splittopskip=2\baselineskip
}

\surroundwithmdframed[style=mymdstyle]{lstlisting}
\newmdenv[style=mymdstyle]{mdlisting}

\begin{document}
\section{Writing a simple linux shell}
\paragraph{Solution}
This is a simple linux shell named \textit{tsh}. It has the following features:
\begin{itemize}
    \item The command line typed by the user should consist of a name and zero or more arguments, 
    all separated by one or more spaces. 
    If name is a built-in command, then \textit{tsh} handles it immediately
    and waits for the next command line. Otherwise, \textit{tsh} assumes that name is the path of an
    executable file, which it loads and runs in the context of an initial child process.
    \item \textit{tsh} does not support pipes (\textbar) or I/O redirection (\textless \space and \textgreater).
    \item Typing \textsf{ctrl-c} (\textsf{ctrl-z}) causes a SIGINT (SIGTSTP) signal to be sent to the current foreground job, 
    as well as any descendents of that job. If there is no foreground job, then the signal will have no effect.
    \item If the command line ends with an ampersand \&, then \textit{tsh} runs the job in the background.
    Otherwise, it runs the job in the foreground.
    \item Each job can be identified by either a process ID (PID) or a job ID (JID), which is a positive integer 
    assigned by \textsf{tsh}. JIDs are denoted on the command line by the prefix \textsf{\%}. For example, \textsf{\%5} 
    denotes JID 5, and \textsf{5} denotes PID 5.
    \item \textit{tsh} supports the following built-in commands:
    \begin{itemize}
        \item \textsf{quit} command terminates the shell.
        \item \textsf{jobs} command lists all background jobs.
        \item \textsf{bg \textless job\textgreater} command restarts \textsf{\textless job\textgreater} 
        by sending it a SIGCONT signal, and then runs it in the background. 
        the \textsf{\textless job\textgreater} argument can be either a PID or a JID.
        \item \textsf{fg \textless job\textgreater} command restarts \textsf{\textless job\textgreater} 
        by sending it a SIGCONT signal, and then runs it in
the foreground. The \textsf{\textless job\textgreater} argument can be either a PID or a JID.
    \end{itemize}
\end{itemize}
The source code of the \textit{tsh} is shown on listing \ref{lst:c1}.
\renewcommand\mylstcaption{\textit{tsh}.c}
\begin{mdlisting}
    \lstinputlisting[caption=\mylstcaption, label=lst:c1]{../tsh.c}
\end{mdlisting}
\end{document}